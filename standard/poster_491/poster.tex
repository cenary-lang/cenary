%%%%%%%%%%%%%%%%%%%%%%%%%%%%%%%%%%%%%%%%%
% baposter Landscape Poster
% LaTeX Template
% Version 1.0 (11/06/13)
%
% baposter Class Created by:
% Brian Amberg (baposter@brian-amberg.de)
%
% This template has been downloaded from:
% http://www.LaTeXTemplates.com
%
% License:
% CC BY-NC-SA 3.0 (http://creativecommons.org/licenses/by-nc-sa/3.0/)
%%%%%%%%%%%%%%%%%%%%%%%%%%%%%%%%%%%%%%%%%

%----------------------------------------------------------------------------------------
%	PACKAGES AND OTHER DOCUMENT CONFIGURATIONS
%----------------------------------------------------------------------------------------

\documentclass[landscape,a1paper,fontscale=0.485]{baposter} % Adjust the font scale/size here

\usepackage{graphicx} % Required for including images
\graphicspath{{figures/}} % Directory in which figures are stored

\usepackage{amsmath} % For typesetting math
\usepackage{amssymb} % Adds new symbols to be used in math mode

\usepackage{booktabs} % Top and bottom rules for tables
\usepackage{enumitem} % Used to reduce itemize/enumerate spacing
\usepackage{palatino} % Use the Palatino font
\usepackage[font=small,labelfont=bf]{caption} % Required for specifying captions to tables and figures

\usepackage{multicol} % Required for multiple columns
\setlength{\columnsep}{1.5em} % Slightly increase the space between columns
\setlength{\columnseprule}{0mm} % No horizontal rule between columns

\usepackage{tikz} % Required for flow chart
\usetikzlibrary{shapes,arrows} % Tikz libraries required for the flow chart in the template

\newcommand{\compresslist}{ % Define a command to reduce spacing within itemize/enumerate environments, this is used right after \begin{itemize} or \begin{enumerate}
\setlength{\itemsep}{1pt}
\setlength{\parskip}{0pt}
\setlength{\parsep}{0pt}
}

\definecolor{lightblue}{rgb}{0.145,0.6666,1} % Defines the color used for content box headers

\begin{document}

\begin{poster}
{
headerborder=closed, % Adds a border around the header of content boxes
colspacing=1em, % Column spacing
bgColorOne=white, % Background color for the gradient on the left side of the poster
bgColorTwo=white, % Background color for the gradient on the right side of the poster
borderColor=lightblue, % Border color
headerColorOne=black, % Background color for the header in the content boxes (left side)
headerColorTwo=lightblue, % Background color for the header in the content boxes (right side)
headerFontColor=white, % Text color for the header text in the content boxes
boxColorOne=white, % Background color of the content boxes
textborder=roundedleft, % Format of the border around content boxes, can be: none, bars, coils, triangles, rectangle, rounded, roundedsmall, roundedright or faded
eyecatcher=true, % Set to false for ignoring the left logo in the title and move the title left
headerheight=0.1\textheight, % Height of the header
headershape=roundedright, % Specify the rounded corner in the content box headers, can be: rectangle, small-rounded, roundedright, roundedleft or rounded
headerfont=\Large\bf\textsc, % Large, bold and sans serif font in the headers of content boxes
%textfont={\setlength{\parindent}{1.5em}}, % Uncomment for paragraph indentation
linewidth=2pt % Width of the border lines around content boxes
}
%----------------------------------------------------------------------------------------
%	TITLE SECTION 
%----------------------------------------------------------------------------------------
%
{\includegraphics[height=4em]{Cenary_Logo.png}} % First university/lab logo on the left
{\bf\textsc{Cenary: A programming language targeting EVM}\vspace{0.5em}} % Poster title
{ Yigit Ozkavci \\\hspace{12pt} Advisor: Can Ozturan } % Author names and institution
{\includegraphics[height=4em]{Boun_Logo.png}} % Second university/lab logo on the right

%----------------------------------------------------------------------------------------
%	INTRODUCTION
%----------------------------------------------------------------------------------------

\headerbox{Introduction}{name=objectives,column=0,row=0}{

Cenary is a programming language designed for Ethereum Virtual Machine. Cenary enables us to write smart contracts and deploy them directly to an Ethereum blockchain.

\vspace{0.3em} % When there are two boxes, some whitespace may need to be added if the one on the right has more content
}

%----------------------------------------------------------------------------------------
%	RESULTS 1
%----------------------------------------------------------------------------------------

\headerbox{Architecture}{name=architecture,column=2,span=2,row=0}{

\begin{center}
\includegraphics[width=0.8\linewidth]{Architecture}
\captionof{figure}{Cenary Architecture}
\end{center}

Cenary compiler first runs the lexer to tokenize cenary source into meaningful pieces. Then parser works through these tokens and creates an abstract syntax tree.\\

This abstract syntax tree directly corresponds to a top level recursive Haskell type called $AnyStmt$. As you might guess, this denotes that an Cenary program is just a big statement, consisting of smaller statements ($Stmt$ type) and expressions ($Expr$ type).\\

You might wonder why we don't have a top level $Stmt$ type, but $AnyStmt$; this is because in Cenary, we pre-process function definitions so that we can call them before their declaration in our program. Hence, $AnyStmt$ expands into $FundefStmt$ and $Stmt$.\\

The magic happens in code generation phase. $Codegen$ module takes ast as input and generates EVM bytecode.
}

%----------------------------------------------------------------------------------------
%	REFERENCES
%----------------------------------------------------------------------------------------

\headerbox{References}{name=references,column=0,above=bottom}{

\renewcommand{\section}[2]{\vskip 0.05em} % Get rid of the default "References" section title
\nocite{*} % Insert publications even if they are not cited in the poster
\small{ % Reduce the font size in this block
\bibliographystyle{unsrt}
\bibliography{sample} % Use sample.bib as the bibliography file
}}

%----------------------------------------------------------------------------------------
%	FUTURE RESEARCH
%----------------------------------------------------------------------------------------

\headerbox{Future Research}{name=futureresearch,column=1,span=2,aligned=references,above=bottom}{ % This block is as tall as the references block

Even though Cenary is currently usable, it still lacks several features such as external type-checker module, recursions, module system and a stable standard library.\\

We plan Cenary to be a pure functional language, hence we need to implement message sending \& receiving in this paradigm. There are good examples such as Elm\cite{elm} and PureScript\cite{purescript} which are pure functional front-end languages managing effects user-inputs in different styles.
}

%----------------------------------------------------------------------------------------
%	CONTACT INFORMATION
%----------------------------------------------------------------------------------------

\headerbox{Contact Information}{name=contact,column=3,aligned=references,above=bottom}{ % This block is as tall as the references block

\begin{description}\compresslist
\item[Web] https://github.com/yigitozkavci
\item[Email] yigitozkavci8@gmail.com
\item[Phone] +90 535 084 76 02
\end{description}
}

%----------------------------------------------------------------------------------------
%	RESEARCH
%----------------------------------------------------------------------------------------

\headerbox{Research}{name=research,column=2,span=2,row=0,below=architecture,above=references}{
Before writing Cenary's compiler, we first needed to explore how EVM works, and how it manages stack \& memory to persist temporary computation steps. \\

There is one and only one source that describes how EVM works, which is called the \textbf{yellowpaper}\cite{yellowpaper} which is a very-formally designed paper that explains every operation that could be executed on EVM in detail.\\

After we figured out how memory is handled, the rest was to learn instruction set of EVM, which itself has an assembly-like instruction set that runs instructions on stack. Compiling Cenary essentially the process of converting cenary source code into EVM bytecode with EVM bytecode being series of bytes, with each one representing an instruction to be executed.
}

%----------------------------------------------------------------------------------------
%	MOTIVATION
%----------------------------------------------------------------------------------------

\headerbox{Motivation}{name=method,column=0,below=objectives,bottomaligned=research}{ % This block's bottom aligns with the bottom of the research block

Smart contracts are widely accepted concept of transferring values without the existence of a middleman. A smart contract lives in the blockchain like any other entity, and is able to receive \& send valuables, with the decision being made via contract's internal state. \\

Every computation on the smart contracts costs gas, a measure for computational power required by the contract; this makes the content of smart contract even more critical since the efficiency of computation directly affects the costs of executing the contract, in real money.
\begin{center}
\includegraphics[width=0.8\linewidth]{smart_contract_infographic}
\captionof{figure}{Smart Contract with Solidity}
\end{center}

Since this is the second term and we continued developing Cenary, which was already a mature programming language, we've made crucial improvements in order to make Cenary more convenient and po
}

%----------------------------------------------------------------------------------------
%	RESULTS 2
%----------------------------------------------------------------------------------------

\headerbox{Language Features}{name=language_features,column=1,bottomaligned=research}{ % This block's bottom aligns with the bottom of the research block

Cenary is a Turing-complete programming language, meaning given enough memory, it can run any given computation.\\

Below is a list of features that Cenary currently supports:
\begin{itemize}
\item Primitives: int, char, bool
\item Array types: int[], char[], bool[]
\item Branching (if-else statements)
\item Loops
\item Functions
\item Arithmetic (+,-,*,/) and comparison (>,==,<) operators
\item Type-checking
\item Parsing error handling
\item Compile-time error handling:
\begin{itemize}
  \item VariableNotDeclared
  \item VariableAlreadyDeclared
  \item VariableNotDefined
  \item TypeMismatch
  \item ScopedTypeViolation
  \item WrongOperandTypes
  \item FuncArgLengthMismatch
  \item ArrayElementsTypeMismatch
  \item NoReturnStatement
\end{itemize}
\end{itemize}
}

Even though Cenary does not have a module system yet, we are capable of implementing a standard library for Cenary, written in Cenary.
%----------------------------------------------------------------------------------------

\end{poster}

\end{document}
